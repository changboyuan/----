\chapter{绪论}
\section{研究背景及意义}
从20世纪七八十年代开始,随着以互联网为代表的网络信息技术的迅速发展,人类社会已经迈入了复杂网络的时代。
人类的生活与生产活动也越来越多地依赖于各种复杂网络的安全可靠和有效运行[*-*]。同时,许多复杂性问题也都可以从复杂网络的角度去研究,因此在国际上形成了非线性科学和复杂性问题的研究热潮[*-*]。
现在,许多发达国家的科学界和工程界都将这个复杂网络这个新兴领域提上了国家科技发展规划的议事日程。在中国,复杂网络作为基础研究也已列入《国家中长期科学和技术发展规划纲要(2006-2020年)》。

系统是由相互作用和相互依赖的若干组成部分结合的具有特定功能的有机整体。而网络和系统通常是密切联系的,如果用节点表示系统的各个组成部分即系统的元素,两个节点之间的连线表示系统元素之间的相互作用,那么网络就为研究系统提供了一种新的描述方式。钱学森将具有自组织、自相似、吸引子、小世界、无标度中部分或全部性质的网络统称为复杂网络[*]。
原则上,任何包含大量组成单元(或子系统)的复杂系统,当我们把构成单元抽象成节点,单元之间的相互作用抽象为边时,都可以当作复杂网络来对其进行研究。用数学的语言来说,复杂网络就是一个有着足够复杂的拓扑结构特征的图。

目前,学者们已在复杂网络的形成机制、网络演化的统计规律和动力学机制、网络结构的稳定性、网络传播以及网络的同步和控制等方面展开研究[*-],取得了大量研究成果。
早期对复杂网络的研究主要集中在规则网络或完全随机网络上。不过,规则网络和完全随机网络都是理想化的模型,而现实世界中的系统往往没有这么理想化,很可能是有序和无序之间的网络(称为小世界网络)[*]。直到1998年,Watts及Strogatz教授提出了经典的Watts-Strogatz(WS)小世界网络[*],该网络在规则网络基础上将每条边以概率$p$进行断边重连后得到。小世界网络中含有大量的局部连边,同时也有少量长程连边,而这些长程连边有效地降低了网络中任意两个节点之间的距离,这正是小世界特性的来源。

随后,研究学者们围绕各种小世界网络的动力学特性展开了研究,包括相应网络演化的统计规律和动力学机制、网络结构的稳定性、网络传播以及网络的同步和控制等,取得了大量研究成果[*-*]。事实上,混沌作为确定性系统中出现的一种类随机现象,揭示了自然界及人类社会中普遍存在的复杂性[*-*]。但由于混沌现象的非线性性以及复杂网络的复杂性,现有文献对小世界网络或者说复杂网络的混沌现象研究较少。以小世界网络方式连接的复杂网络会具有怎样的动力学特性尤其是混沌现象,是非常值得研究的一个问题。与此同时,同步现象作为复杂网络最重要的动力学特性之一普遍存在于各类复杂网络中,对复杂动态网络的同步控制是复杂网络研究和应用的关键环节[6-8]。对小世界复杂网络的同步现象进行研究,无疑也具有重大的现实研究意义和广阔的应用前景,

此外,混沌信号由于其类噪声特性和长期不可预测性,已被广泛研究并用于保密通信、雷达信号处理、信号检测等诸多领域。
但在实际情况下,混沌信号总是会被噪声污染,而混沌信号具有非周期、宽带频谱等特性,一些现有的信号复原方法在处理混沌信号时难以获得理想的效果。
因此,研究噪声污染下混沌信号的重构技术具有重要意义,有效的混沌重构技术也将大大提高各种应用的性能。

以压缩感知为典型代表的稀疏理论提出:稀疏的或具有稀疏表达的有限维数的信号可以利用远少于奈奎斯特采样数量的线性、非自适应的测量值无失真地重建出来。
该理论一经提出,便在信息论、信号/图像处理、医疗成像、射电天文、模式识别、光学/雷达成像和信道编码等诸多领域引起广泛关注[*-*]。
在信号处理领域,信号的稀疏重构理论仍是一个较新的研究方向,近年来,学者们在该方向已取得了一些显著的研究成功,但很多问题仍需进一步研究。事实上,该理论的应用前提是能够对需处理的信号进行直接或间接的稀疏建模,合理地选取具有等距约束等限制条件的采样矩阵以及提出更高精度、更低复杂度或对噪声更鲁棒的后端恢复算法[*-*]。此外,含噪混沌信号的稀疏重构问题尚未有完善的研究。因此,能否将稀疏理论用于复杂网络带噪混沌信号的重构具有重要意义,有效的混沌重构算法也将大大提高混沌信号各种应用的性能。

\section{研究现状}
\subsection{复杂网络混沌、同步动力学研究现状}
20世纪中叶以来,网络系统的科学研究日益受到重视。但是,直至世纪末,关于网络结构的假设基本上是两个极端:完全规则的结构或者完全随机的拓扑。
例如,在20 世纪的非线性动力学研究中,绝大多数网络动态系统模型均假设网络具有规则而且固定的拓扑结构,把重点放在由节点的非线性动力学行为所产生的复杂性,
如斑图的涌现和时空混沌的产生等。典型的例子包括耦合映象格子和非线性网络。采用简单的規则结构 的主要好处是使人们可以集中精力研究节点的复杂动力学行为对整个网络复杂
性的影响,并且规则的网络结构也便手用集成电路实现。而20世纪50年代末由两位匈牙利数学家Erdos和Renyi建立的随机图理论被公认为是上开创复杂网络拓扑结构的系统性研究,至今仍然在网络分析中起着重要作用。\par
最早的小世界网络动力学研究在1998年Watts及Strogatz的文章中给出,他们利用相应的小世界模型模拟传染病在人群中传播,
研究表明相较于规则网络,小世界网络的传播能力明显要快得多。随后,研究学者们围绕各种小世界网络的动力学特性展开了研究[*-*]。
2001年,Zhuo Gao研究了小世界网络的随机共振现象,
发现小世界网络的随机共振效应相比与普通网络有所增强[*]。2002年,H. Hong等探讨了小世界网络同步性并发现随着重连概率的增大,
小世界网络中各个振子间同步性显著提高。2001年,Xin-She Yang对一个非线性时滞混沌小世界网络进行研究,
发现小世界混沌网络的传播要比规则网络速度更快。2012年,Li Ning提出了一个基于小世界网络的离散复杂网络,并研究其分叉和混沌等动力学特性后发现相较于规则网络,
小世界网络的混沌现象在适当的参数下会得到控制。\par
21世纪是复杂性和网络化的世纪。从20世纪七八十年代开始,在国际上形成了非线性科学和复杂性问题的研究热潮。尤其是20世纪90年代以来,
人类已经生活在一个充满各种各样复杂网络的世界中,许多复杂性问题都可以从复杂网络的角度去研究。
随着以互联网为代表的网络信息技术的迅速发展,人类社会已经迈入了复杂网络时代。人类的生活与生产活动越来越多地依赖于各种复杂网络系统安全可靠和有效的运行。
作为一个跨学科的新兴领域,“网络科学与工程”已经逐步形成并获得了迅猛发展。现在,许多发达国家的科学界和工程界都将这个新兴领域提上了国家科技发展规划的议事日程。
在中国,复杂系统包括复杂网络作为基础研究也已列入《国家中长期科学和技术发展规划纲要(2006-2020年)》。
\subsection{信号的稀疏重构研究现状}
稀疏重构是目前最优化领域中非常热门的研究课题,最早是由美国科学院院士David Donoho等人于1998年提出来的[*],它的本质思想是结合解的稀疏性结构来构建数学模型,克服欠定线性反问题的不适定性,进而提升模型的稳定性和准确性。2005年,数学家Emmanuel Candès与陶哲轩给出了稀疏重构理论的数学理论,证明在已知信号的稀疏性情况下,稀疏重构优化模型能够利用极少数的采样数(显著优于奈奎斯特采样定理)来重建原信号,奠定了稀疏优化模型的理论根基[*]。在过去的10多年中,稀疏优化模型吸引了学术界与业界的大量关注,并且在诸如压缩感知、图像科学、机器学习、统计建模和基因组学数据分析等很多领域都取得了成功的应用。

目前已有的稀疏重构求解算法主要可归为三大类: 

(1)	凸优化算法:这类方法是将非凸问题转化为凸问题求解以找到信号的逼近,如基追踪(Basis Pursuit, BP)f算法,梯度投影方法等。这类算法的
核心思想是将NP-hard的0范数优化问题转换为可解的凸优化问题,即1范数问题,随后使用一般的凸优化方法来解决。这种做法需要具备一定条件与限制,
并且算法复杂度高,迭代时间久。

(2)	贪婪算法:这类算法是通过每次迭代时选择一个局部最优解来逐步逼近原始信号,典型的贪婪算法是匹配追踪算法(MP),通过一定的选择标准来确定
信号的支撑集,逐步提高其稀疏度。最常见的是正交匹配追踪法OMP,又发展出了StOMP,CoSaMP等改良算法。
总的来说,贪婪类稀疏重构算法速度快, 然而需要的测量数据多且精度低。

(3)	组合算法:这类方法要求信号的采样支持通过分组测试快速重建,如代表性方法稀疏贝叶斯学习方法(SBL)。该类方法位于前两者之间,具有强于
优化算法的适应性与优于贪婪算法的最终结果。
\section{本文主要工作和论文结构}
本文在经典Duffing振子的基础上,提出一个以WS小世界网络方式进行连接的Duffing复杂网络(简称Duffing-WS型小世界网络),并研究该复杂小世界网络的混沌现象,分析复杂系统耦合强度、重连概率、邻接度等参数对其混沌特性的影响规律;同时考虑采用稀疏重构理论建立重构算法还原小世界网络生成的带噪混沌信号。

本文的主要工作和论文结构如下:

第1章介绍本文的研究背景及意义和研究现状。

第2章介绍本文工作涉及内容所需的基础理论。首先介绍复杂网络及小世界网络的基本概念和性质,接着介绍非线性动力系统的混沌以及同步基本理论,最后梳理稀疏重构的基本理论,包括稀疏重构的基本数学模型、实现稀疏重构所需基本条件,以及稀疏重构常见数值算法。

第3章针对所提Duffing-WS型小世界网络,研究该网络的混沌、同步等基本动力学特性。

(1) 首先在经典Duffing振子的基础上,提出一个以WS小世界网络方式进行连接的Duffing复杂网络(简称Duffing-WS型小世界网络),利用变分法推导Duffing-WS型小世界网络的最大李雅普诺夫指数表达式,以庞加莱截面分岔图和李雅普诺夫指数为工具研究该复杂网络的混沌现象,通过微分方程数值解法进行数值仿真,使用LE指数衡量系统的混沌程度与振幅范围,分析小世界网络重连度,重连概率和耦合强度对此复杂网络混沌现象的影响。相应的仿真分析表明,本文所提Duffing-WS小世界网络的各个粒子输出也呈现出小尺度周期运动、倍周期分岔、混沌和大尺度周期运动等多种状态,其混沌的参数范围较单个Duffing方程更为复杂。网络重连度,重连概率和耦合强度等参数对其混沌区域的影响也较传统规则网络有明显不同。

(2)接着,同样基于变分法给出本文所提Duffing-WS小世界网络(*)变法方程所对应的主稳定函数,并由此分析Duffing-WS小世界网络的同步性。分析表明,本文所提Duffing-WS小世界网络的同步性主要受到小世界网络连接拓扑矩阵Laplacian特征值和耦合强度等参数影响。

第4章基于稀疏重构理论提出了一种针对被噪声污染混沌信号的重构算法,仿真实验表明, 该方法能较为稳健地恢复受噪声干扰的Duffing-WS型小世界网络输出的带噪混沌信号,不仅较具有更高的输出信噪比, 而原始信号的混沌特性也能得到较大程度的恢复,这是一般稀疏重构算法不具有的。

第5章总结全文。