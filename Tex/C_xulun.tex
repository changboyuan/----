\chapter{绪论}
\section{研究背景及意义}
从20世纪七八十年代开始,在国际上形成了非线性科学和复杂性问题的研究热潮。许多复杂性问题都可以从复杂网络的角度去研究。
钱学森将具有自组织、自相似、吸引子、小世界、无标度中部分或全部性质的网络统称为复杂网络。
原则上,任何包含大量组成单元(或子系统)的复杂系统,当我们把构成单元抽象成节点,单元之间的相互作用抽象为边时,都可以当作复杂网络来对其进行研究。
早期对复杂网络的研究主要集中在规则网络或完全随机网络上。
不过,规则网络和完全随机网络都是理想化的模型,而现实世界中的系统往往没有这么理想化,
很可能是有序和无序之间的网络(称为小世界网络)。早在1998年,Watts及Strogatz教授提出了经典的WS小世界网络,
该网络在规则网络基础上将每条边以概率p进行断边重连后得到。
小世界网络中含有大量的局部连边,同时也有少量长程连边,而这些长程连边有效地降低了网络中任意两个节点之间的距离,
这正是小世界特性的来源。
\section{研究现状}
最早的小世界网络动力学研究在1998年Watts及Strogatz的文章中给出,作者利用相应的小世界模型模拟传染病在人群中传播,
研究表明相较于规则网络,小世界网络的传播能力明显要快得多。2001年,Zhuo Gao研究了小世界网络的随机共振现象,
发现小世界网络的随机共振效应相比与普通网络有所增强。2002年,H. Hong等探讨了小世界网络同步性并发现随着重连概率的增大,
小世界网络中各个振子间同步性显著提高。2001年,Xin-She Yang对一个非线性时滞混沌小世界网络进行研究,
发现小世界混沌网络的传播要比规则网络速度更快。2012年,
Li Ning提出了一个基于小世界网络的离散复杂网络,并研究其分叉和混沌等动力学特性后发现相较于规则网络,
小世界网络的混沌现象在适当的参数下会得到控制。

此外,混沌信号由于其类噪声特性和长期不可预测性,已被广泛研究并用于保密通信、雷达信号处理、信号检测等诸多领域。
但在实际情况下,混沌信号总是会被噪声污染,而混沌信号具有非周期、宽带频谱等特性,一些现有的信号复原方法在处理混沌信号时难以获得理想的效果。
因此,研究噪声污染下混沌信号的重构技术具有重要意义,有效的混沌重构技术也将大大提高各种应用的性能。以压缩感知为典型代表的稀疏理论提出:
稀疏的或具有稀疏表达的有限维数的信号可以利用远少于奈奎斯特采样数量的线性、非自适应的测量值无失真地重建出来。
该理论一经提出,便在信息论、信号/图像处理、医疗成像、射电天文、模式识别、光学/雷达成像和信道编码等诸多领域引起广泛关注。
在信号处理领域,信号的稀疏重构理论仍是一个较新的研究方向,
近年来,学者们在该方向已取得了一些显著的研究成功,但很多问题仍需进一步研究。事实上,该理论的应用前提是能够对需处理的信号进行直接或间接的稀疏建模, 
合理地选取具有等距约束等限制条件的采样矩阵以及提出更高精度、更低复杂度或对噪声更鲁棒的后端恢复算法。
而能否将稀疏理论用于复杂系统带噪混沌信号的重构也是一个非常值得探讨的问题。

以压缩感知为典型代表的稀疏重构理论是一个较新的研究方向,一经提出便在信息论、信号/图像处理、医疗成像、射电天文、模式识别、光学/雷达成像和信道编码等诸多领域引起广泛关注[12,13]。
稀疏理论的应用前提是能够对信号进行直接或间接的稀疏化, 合理地选取具有等距约束等限制条件的采样矩阵以及提出高精度、低复杂度和抗噪声的重构算法[14]。
含噪混沌信号的稀疏重构问题尚未有完善的研究。因此,能否将稀疏理论用于复杂网络带噪混沌信号的重构具有重要意义,有效的混沌重构算法也将大大提高混沌信号各种应用的性能。
\section{本文主要工作和论文结构}
本文提出一个以WS小世界网络方式进行连接的Duffing复杂网络(简称Duffing-WS型小世界网络),并研究该复杂小世界网络的混沌现象,分析复杂系统耦合强度、重连概率、邻接度等参数对其混沌特性的影响规律;同时考虑采用稀疏重构理论建立重构算法还原小世界网络生成的带噪混沌信号。
利用变分法推导Duffing-WS型小世界网络的最大李雅普诺夫指数表达式,以庞加莱截面分岔图和李雅普诺夫指数为工具研究该复杂网络的混沌现象,通过微分方程数值解法进行数值仿真,使用LE指数衡量系统的混沌程度与振幅范围,分析小世界网络重连度,重连概率和耦合强度对此复杂网络混沌现象的影响。
现有文献主要研究复杂网络的同步、传播和共振等动力学特性,对复杂网络的混沌研究较少。本文针对一个以WS小世界网络方式进行连接的Duffing复杂网络,研究其混沌特性,并基于优异的并行处理计算架构相应的数值仿真算法分析各个参数对混沌的影响。
已针对研究领域进行了必要的理论调研和学习了相关基础理论,针对提出的模型利用变分法推导Duffing-WS型小世界网络的最大李雅普诺夫指数表达式;此外具有并行处理能力的计算机,高性能程序开发经验,能够基于优异的并行处理计算架构给出相应的数值仿真算法。
理清耦合强度、重连概率、邻接度等参数的变化对系统混沌的影响,找到控制复杂网络混沌现象的关键;基于稀疏理论给出复杂系统带噪混沌信号的重构算法。
