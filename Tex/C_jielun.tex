\chapter{结论}
复杂网络广泛存在于各个科学领域,对复杂网络的研究目前已成为非线性科学和复杂性问题领域的一个研究热点。复杂网络的各个节点具有自身的动力学,而这些节点由于网络拓扑结构的不同而演化出不同的群体行为,可能收敛于平衡点、周期轨或者混沌吸引子。混沌作为复杂网络重要的动力学特性,研究其产生机理和对参数的依赖性,构建相应混沌信号在噪声干扰下的重构方法,对复杂网络的混沌控制和混沌信号的各种应用都具有重要的意义。
\par 本文生成一种新的Duffing-WS型小世界网络模型,通过变分法计算出其最大李雅普诺夫指数表达式作为衡量网络混沌的标准,分析不同参数对网络混沌的影响。同时运用稀疏采样还原算法成功压缩且高精度还原了带噪声的混沌信号,并且对比了不同算法和在混沌信号情形下的性能。
我们发现,Duffing-WS小世界网络具有比单个Duffing-方程更为复杂的混沌特性,而系统重连概率、重连度以及耦合强度对系统混沌区域的影响也有别于传统规则网络:
和传统的规则网络不同,网络耦合强度$\varepsilon$对混沌的影响并不是单调的,当网络重连度K较小时,耦合强度的增强反而会促进系统的混沌现象;只有在重连度K增大到一定程度之后,较强的耦合强度才会对系统混沌起到控制效果;但是重连度K足够大以后,系统混沌
网络重连度K在不同重连概率p下对混沌有明显的影响,对于规则网络(p = 0)和小世界网络(0< p <1),足够大的重连度会抑制系统混沌,较小的重连度则促进系统的混沌运动;相比前面两种网络,完全随机网络(p = 1)的混沌区域则更大,重连度K对其混沌区域的影响也呈现非单调性,随着重连度K的增加,其混沌区域先增加后减小。
网络重连概率p对复杂网络混沌区域的影响不明显。\par
本文同时对小世界网络模型的同步性进行了初步研究,给出了重连概率与网络同步性的分析,在一定情形下,重连概率极大的促进了
混沌系统的同步,也就是说,随着复杂系统随机性的增强,整个系统会更容易同步。最后本文针对混沌信号进行稀疏重构算法的适应性研究,得出混沌信号对
两种算法的适应性分析,CoSaMP算法在混沌信号的稀疏重构能力极大强于OMP算法,这说明混沌信号的稀疏重构特性值得探讨。